	\documentclass[11pt, a4paper, spanish]{article}

\usepackage{alltt}

%%%%%%%%%% COMIENZO DEL PREAMBULO %%%%%%%%%%

%Info sobre este documento
\author{Mart\'n Cammi}
\title{Trabajo Pr\'actico de Algoritmos y Estucturas de datos III}

%\usepackage{infostyle}                                                  % provee un look & feel similar a un documento Word
\usepackage[top=2.5cm, bottom=2.5cm, left=2.5cm, right=2.5cm]{geometry}  % m\'argenes
\usepackage[ansinew]{inputenc}                                           % permite que los acentos del estilo \'a\'e\'i\'o\'u salgan joya
\usepackage[spanish, activeacute]{babel}                                 % idioma espaniol, acentos f\'aciles y deletreo de palabras
\usepackage{indentfirst}                                                 % permite indentar un parrafo a mano
\usepackage{caratula}                                                    % incluye caratula est\'andar
\usepackage{graphicx}                                                    % permite insertar gr\'aficos
\usepackage{color}                                                       % permite el uso de colores en el documento
\usepackage{amssymb}
\usepackage[pdfcreator={TexLive!, LaTeX2e con TeXnicCenter},
			pdfauthor={Grupo: "1"},
			pdftitle={Algoritmos III - Trabajo Pr\'actico 2},
			pdfsubject={Trabajo Pr\'actico 1},
			pdfkeywords={Prismas, Esferas, Punto de corte, puente, pizza},
			pdfstartview=FitH,            % Fits the width of the page to the window
			bookmarksnumbered,            % los bookmarks numerados se ven mejor...
			colorlinks,                   % links con bellos colores
			linkcolor=magenta]            % permite cambiar el color de los links
			{hyperref}                    % Permite jugar con algunas cosas que aparecer\'an en el PDF final

%RESTAURAR

\usepackage{algorithm}							% Permite tabular un codigo
\usepackage{algorithmic}
%\input{spanishAlgorithmic} % mi archivo de traducci\'on			

%\selectlanguage{spanish}

%\selectlanguage{spanish}
\linespread{1.3}                    % interlineado equivalente al 1.5 l\'ineas de Word...
\pagestyle{myheadings}              %encabezado personalizable con \markboth{}{}
\markboth{}{Algoritmos III  - Trabajo Pr\'actico 2 - Cammi, Garbi, Kretschmayer}
\headsep = 30pt                     % separaci\'on entre encabezado y comienzo del p\'arrafo

%\addtolength{\oddsidemargin}{-2cm}	% configuracion IDEAL!!!
%\addtolength{\textwidth}{4cm}
%\addtolength{\textheight}{2cm}

% macro 'todo' para To-Do's
\def\todo#1{\textcolor{red}{#1}}

% Macro 'borde' para un texto con borde
\newsavebox{\fmbox}
\newenvironment{borde}[1]
{\begin{lrbox}{\fmbox}\begin{minipage}{#1}}
{\end{minipage}\end{lrbox}\fbox{\usebox{\fmbox}}\\[10pt]}

%%%%%%%%%% FIN DEL PREAMBULO %%%%%%%%%%

\begin{document}

\materia{Algoritmos y Estucturas de datos III}
\submateria{Segundo Cuatrimestre de 2011}
\titulo{Trabajo Pr\'actico 3}
\subtitulo{Cartero Chino en grafo mixto.}
\grupo{Grupo: `1'}
\integrante{Cammi, Mart\'in}{676/02}{martincammi@gmail.com}
\integrante{Garbi, Sebasti\'an}{179/05}{garbyseba@gmail.com}
\integrante{Kretschmayer, Daniel}{310/99}{daniak@gmail.com}

\maketitle

\thispagestyle{empty}

\tableofcontents

\newpage


\textbf{Ejecuci\'on del TP}
\label{sec:ejecucion}

	\subsection{Lenguaje utilizado}
		
		El lenguaje utilizado para el trabajo pr\'actico ha sido \emph{Java}, compilando con la versi\'on 1.5 de la Virtual Machine.
		
		El trabajo se acompa\~{n}a con los fuentes de la soluci\'on que puede importarse en IDE de Eclipse o ejecutarse desde l\'inea de comandos.

	\subsection{Como ejecutar el TP}
	
	\textbf{\underline{Desde l\'inea de comandos}}
	\begin{itemize}
			\item{Posicionarse en el directorio Algo3Tp2}
			\item{Copiar all\'i el archivo de entrada para el problema i, por ejemplo Ej1.in}
			\item{Ejecutar el comando: \emph{java -cp ./bin problema1.Ej1}}
	\end{itemize}
	Esto generar\'a el archivo Ej1.out con la soluci\'on en el mismo directorio Algo3Tp2.

	\textbf{\underline{Desde el Eclipse}}
	
	Primero importaremos el proyecto:	
	
	\begin{itemize}
			\item{Seleccionar File $\Rightarrow$ Import.}
			\item{Seleccionar General $\Rightarrow$ Existing Projects into Workspace $\Rightarrow$ Next.}
			\item{Seleccionar el directorio llamado Algo3Tp2.}
			\item{Finish.}
	\end{itemize}
	
	Desde la vista de \textbf{Package Explorer} bajo el paquete \textbf{src} aparecer\'an tres paquetes m�s y dentro de cada uno de ellos los siguientes archivos de java:\\

	\begin{center}
		%\includegraphics[scale=0.65]{others/packageExplorer.png}\\
		COLOCAR IMAGEN CORRECTA
	\end{center}

\newpage

	Para ejecutar un problema:

	\begin{itemize}
			\item{Posicionarse en el directorio Algo3Tp2}
			\item{Copiar en el directorio Algo3Tp2 el archivo de entrada para el problema i, por ejemplo Ej1.in}
			\item{Con bot\'on derecho Run As $\Rightarrow$ Java Application. Se ejecutar\'a el problema seleccionado.}
	\end{itemize}
	Esto generar\'a el archivo Ej1.out con la soluci\'on en el mismo directorio Algo3Tp2.

\newpage

\section{Situaciones de la vida real}
\section{Heuristica Constructiva}
\section{Heuristica de busqueda Local}
Nuestro algoritmo de busqueda local se basa en eliminar aristas y arcos agregados para igualar los grados de salida a los de entrada de algunos nodos y cambiarlos por un conjunto de aristas que cumplen la misma propiedad pero cuya suma es menor.
Para ello 
\section{Metaheuristica Grasp}

%\label{sec:problema1}
%\subsection{Introducci\'on}
	

\end{document}
